
%%%%%%%%%%%%%%%%%%%%%%%%%%%%%%%%%%%%%%
%%   End of Harvard style referencing setup %%

%%%%%%%%%%%%%%%%%%%%%%%%%%%%%%%%%%%%%%%%%
% University Assignment Title Page 
% LaTeX Template
% Version 1.0 (27/12/12)
%
% This template has been downloaded from:
% http://www.LaTeXTemplates.com
%
% Original author:
% WikiBooks (http://en.wikibooks.org/wiki/LaTeX/Title_Creation)
%
% License:
% CC BY-NC-SA 3.0 (http://creativecommons.org/licenses/by-nc-sa/3.0/)
% 
% Instructions for using this template:
% This title page is capable of being compiled as is. This is not useful for 
% including it in another document. To do this, you have two options: 
%
% 1) Copy/paste everything between \begin{document} and \end{document} 
% starting at \begin{titlepage} and paste this into another LaTeX file where you 
% want your title page.
% OR
% 2) Remove everything outside the \begin{titlepage} and \end{titlepage} and 
% move this file to the same directory as the LaTeX file you wish to add it to. 
% Then add \input{./title_page_1.tex} to your LaTeX file where you want your
% title page.
%
%%%%%%%%%%%%%%%%%%%%%%%%%%%%%%%%%%%%%%%%%
%\title{Title page with logo}
%----------------------------------------------------------------------------------------
%	PACKAGES AND OTHER DOCUMENT CONFIGURATIONS
%----------------------------------------------------------------------------------------

\documentclass[12pt]{article}
\usepackage[british]{babel}
\usepackage{csquotes}
\usepackage[utf8]{inputenc}
\usepackage{amsmath}
\usepackage{graphicx}
\usepackage[colorinlistoftodos]{todonotes}
\usepackage{caption}
\usepackage{subcaption}
\usepackage[toc]{appendix}
\usepackage{titling}

% Harvard style setup from http://tex.stackexchange.com/questions/134258/harvard-style-bibliography-with-biblatex-almost-but-not-quite
\usepackage[
firstinits=true, % render first and middle names as initials
useprefix=true,
maxcitenames=3,
maxbibnames=99,
style=authoryear,
dashed=false, % re-print recurring author names in bibliography
backend=bibtex,
]{biblatex}

\renewcommand*{\bibinitdelim}{}

\usepackage{xpatch}
\xapptobibmacro{date+extrayear}{\nopunct}{}{}

% Use single quotes around titles:

\DeclareNameAlias{author}{last-first}
\renewcommand*{\mkbibnamefirst}[1]{{\let~\,#1}} % insert thin spaces between author initials
\renewcommand*{\nameyeardelim}{\addcomma\addspace} % insert a comma between author and year in-text citations
\renewcommand*{\newunitpunct}{\addcomma\addspace} % comma as separator in bibliography, not full stop
\setlength\bibitemsep{1.5\itemsep} % increase spacing between entries in bibliography
\renewbibmacro{in:}{} % remove 'in:' preceding article title

% Place volume number within parentheses:
\renewbibmacro*{volume+number+eid}{
    \printfield{volume}
    \setunit*{\addnbspace}% NEW (optional); there's also \addnbthinspace
    \printfield{number}
    \setunit{\addcomma\space}
    \printfield{eid}}
\DeclareFieldFormat[article]{number}{\mkbibparens{#1}}


\begin{document}

\begin{titlepage}

\newcommand{\HRule}{\rule{\linewidth}{0.5mm}} % Defines a new command for the horizontal lines, change thickness here

\center % Center everything on the page
 
%----------------------------------------------------------------------------------------
%	HEADING SECTIONS
%----------------------------------------------------------------------------------------

%----------------------------------------------------------------------------------------
%	LOGO SECTION
%----------------------------------------------------------------------------------------

\includegraphics[width=9cm,keepaspectratio]{abertay_logo.jpeg}\\[2cm] % Include a department/university logo - this will require the graphicx package

\textsc{\Large Honours Project Proposal}\\[0.5cm] % Major heading such as course name
\textsc{\large CMP401}\\[0.5cm] % Minor heading such as course title

%----------------------------------------------------------------------------------------
%	TITLE SECTION
%----------------------------------------------------------------------------------------

\HRule \\[0.4cm]
{ \huge \bfseries An evaluation of the Visibility Buffer rendering technique in
real-time graphics applications.}\\[0.4cm] % Title of your document
\HRule \\[1.5cm]
 
%----------------------------------------------------------------------------------------
%	AUTHOR SECTION
%----------------------------------------------------------------------------------------

\begin{minipage}{0.4\textwidth}
\begin{flushleft} \large
\emph{Author:}\\
Alberto \textsc{Taiuti} % Your name
\end{flushleft}
\end{minipage}
~
\begin{minipage}{0.4\textwidth}
\begin{flushright} \large
\emph{Supervisor:} \\
Dr. Paul \textsc{Robertson} % Supervisor's Name
\end{flushright}
\end{minipage}\\[2cm]

% If you don't want a supervisor, uncomment the two lines below and remove the section above
%\Large \emph{Author:}\\
%John \textsc{Smith}\\[3cm] % Your name

%----------------------------------------------------------------------------------------
%	DATE SECTION
%----------------------------------------------------------------------------------------

{\large \today}\\[2cm] % Date, change the \today to a set date if you want to be precise



\vfill % Fill the rest of the page with whitespace

\end{titlepage}


%\addbibresource{RW_6648_1300250-RefList.bib}
%\addbibresource{bibliography.bib}
%\addbibresource{prof_proj_3rd_year.bib}

% Table of contents page
%\pagenumbering{arabic}
%\tableofcontents
%\newpage

\begin{abstract}
Over the past couple of years
\end{abstract}

\section{Introduction}
Computer graphics is a field which is constantly in evolution. New technologies
and techniques are developed every day and there is a constant push for
innovation both in terms of performance and of visual quality.
Over the last decade Deferred Rendering has become one of the primary choices of
rendering technique for game developers due to its advantages over Forward
Rendering and there have been other major publications showcasing novel
techniques which propose to improve over one or more aspects of Deferred or
Forward Rendering, with one example of these being Forward+ rendering.\\

As this area of research is extremely active, new techniques are
published every year. However, one specific area of improvement being
researched is how to reduce
the memory footprint of the rendering framework used by an application so
that more GPU
memory is available for other purposes, such as storing textures or general
usage buffers.
This is especially relevant when put in the context of the games industry's
push to
achieve real-time 4K high-resolution rendering in computer games, where
because of the high pixel density of the images being rendered, a large amount
of the memory available on GPUs is used for storing the images used by
the rendering engine, leaving less space left for the texture maps used by the
3D models which also need to be of higher resolution to avoid aliasing with the
higher resolution render targets.
More so, with the advent of Virtual Reality headsets and their high rate of
adoption by customers, it becomes even more important to explore how to
reduce the memory footprint of the available rendering techniques given that
to render a 3D VR environment, the whole scene needs to be rendered twice and
the
results of the rendering stored in memory for both the left and right eye,
basically doubling the amount of memory required by the renderer.\\

A solution to this problem is to simply increase the memory available on the
GPU, which some hardware vendors have implemented. However this approach
requires the end users to buy a new GPU and reduces the reachable audience
if your game or 3D application requires such hardware.\\

Another solution is to develop and 
assess new software approaches which allow to improve the memory efficiency of
3D renderers with the currently available hardware. On this side,
there has recently been an example of such technique being brought forward,
named in various ways by different authors but more generally referred to as
\emph{The Visibility Buffer}.\\
The Visibility Buffer rendering technique proposes to reduce the number of
intermediary image buffers required to render a scene at the expense of more
complex real-time calculations on the GPU, thus reducing the overall memory
footprint of the renderer implementation.
However, on its own this is not appealing due to the more complex work
done by the shaders which makes it potentially too slow for real-time
applications and hence research is going towards mixing the basic
Visibility Buffer with other methods in order to improve its real-time 
rendering speed.
This technique is very novel and is yet to be fully explored, which
makes it a good candidate for an in-depth research on its feasibility as a
real-time rendering method when compared to consolidated ones such as
Forward, Forward+ and Deferred Rendering and whether its advantages in memory
footprint are met by a decent rendering speed performance.\\

Hence the research question for this project is:
\setlenght{\parindent}{3em}
\emph{How Visibility Buffer technique compare to current real-time
rendering techniques and what techniques exist to improve its real-time
rendering performance?}



many of them set to improve the currently
available ones, whereas only a few actually propose a new way of achieving
real-time 3D rendering using the present software and hardware.


\end{document}
